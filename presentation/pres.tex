%
%	LaTeX - SCHABLONE fuer die VORTRAGSPRÄSENTATION
%
%	\documentclass[hidesubsections,inrow]{beamer}
\documentclass[hidesubsections,compress]{beamer}
%
% What to output:
%       [notes], [notesonly], [trans], [handout]
% Font family, size selection:
%       [sans], [serif], [mathsans], [mathserif]
%       {smaller,bigger}
% General color trend:
%       {red,blue,grey,brown}
% Topology options:
%       {hide,shade}subsections compress
%       slides{centered,top}

%usepackage{beamerthemesidebardarktab}
%	\usepackage{beamertemplates}		%% no longer supported
%\usepackage[latin1]{inputenc}
%\usepackage[german]{babel}


\usepackage[utf8]{inputenc}
\usepackage[ngerman]{babel}


\usepackage{amsmath,amssymb}
\usepackage{graphicx}
\usepackage{textcomp}
\usepackage{hyperref}
\usepackage{listings}
\usepackage{caption}
\usepackage{subcaption}
\usepackage{multimedia}
\usepackage{pdfpcnotes}
\usepackage{sansmathaccent}
\pdfmapfile{+sansmathaccent.map}

%\makeatletter
%\let\@noitemerr\relax
%\makeatother

\defbeamertemplate{description item}{align left}{\insertdescriptionitem\hfill}


\usetheme{Singapore}
%\usecolortheme{beetle}
\usefonttheme[onlysmall]{structurebold}

\newcommand\lvtyp{SEMINAR}
\newcommand\lvname{Rechnersehen}
\newcommand\lvinst{Computervision Group · FSU Jena}

%
%	HIER WERDEN TITEL REFERENT UND DATUM EINGETRAGEN
%
\newcommand\svthema{3D-Deskriptoren für Aufgaben der Rekonstruktion und Objekterkennung}
\newcommand\svperson{Christian Lengert}
\newcommand\svdatum{20.~Juni 2017}

\title{ \textbf{\textcolor{blue}{\svthema}} }
\author{ \emph{\textcolor{red}{\svperson}} }
\institute{ {\lvtyp} \emph{"`\lvname"'} }
\date{ \tiny \lvinst \\ \svdatum }

\setcounter{tocdepth}{2}
\begin{document}
   \frame { \titlepage }

   \frame{
      \frametitle{Vortragsgliederung}
 		\begin{small}
			\tableofcontents%[subsectionstyle=hide]
    	\end{small}
   }


   \frame{
   \frametitle{Unique Signatures of Histograms for Local Surface Description}
			%\begin{center}
			\begin{itemize}
				\item Frederico Tombari  
				\item Samuele Salti
				\item Luigi Di Stefano
				\item Universitiy of Bologna
				\item 2010
			\end{itemize}
			%\end{center}
			\includegraphics[width=\textwidth]{pape.jpg}
   }
   \section {Einführung}
%\frame{ \tableofcontents[currentsection] }
%!TEX root = ../../main.tex
\frame{
\subsection{3D-Deskriptor}
\frametitle{3D-Deskriptor?}
	\setbeamertemplate{description item}[align left]
	\begin{description}
		\item[Eingabe] Tiefeninformationsgewinnug $\rightarrow$ \textbf{3D-Punktwolke}
		\item[Verarbeitung] Berechnung, welche Punkt und seine Nachbarschaft betrachtet.
		\item[Ausgabe] Möglichst eindeutige Beschreibung
	\end{description}

	    \begin{figure}
        \centering
        \begin{subfigure}[b]{0.3\textwidth}
            \includegraphics[width=\textwidth]{topics/intro/n2d.png}
            \caption{2D-Nachbarschaft}
            \label{fig:2dn}
        \end{subfigure}
        ~~~
        \begin{subfigure}[b]{0.3\textwidth}
            \includegraphics[width=\textwidth]{topics/intro/n3d.png}
            \caption{3D-Nachbarschaft}
            \label{fig:3dn}
        \end{subfigure}
    \end{figure}


}

\subsection{Objekterkennung}
\frame{
	\frametitle{Objekterkennung}
	Finde bekanntes Objekt in einer Szene:
	\begin{enumerate}
		\item Berechne Deskriptoren für Modell
		\item Berechne Deskriptoren für Szene
		\item Vergleiche berechnete Deskriptoren
	\end{enumerate}


		 \begin{figure}
        \centering
        \begin{subfigure}[b]{0.45\textwidth}
            \includegraphics[width=\textwidth]{topics/intro/matchingScene.png}
            \caption{Finde Objekt}
            \label{fig:find}
        \end{subfigure}
        ~~~
        \begin{subfigure}[b]{0.45\textwidth}
           \includegraphics[width=\textwidth]{topics/intro/objRec1.jpg}
            \caption{Bestimme Lage.}
            \label{fig:3dn}
        \end{subfigure}
    \end{figure}

	\centering
		
}

\subsection{Rekonstruktion}
\frame{
	\frametitle{Rekonstruktion}

\begin{flushleft}
\setbeamertemplate{description item}[align left]
            \begin{description}
            	\item[Genauigkeit] von verwendeten Verfahren nicht ausreichend.
            	\item[Textur] ist von entscheidender Bedeutung
            	\item[Interpolation] von gewonnenen Bildpunkten.
            	\item[Besseres] Modell der Umgebung
			\end{description}

		\begin{itemize}
			\item Automatisierte Baumaschinen
			\item Automatische Landung von Raumsonden
		\end{itemize}
\end{flushleft}

\setbeamertemplate{description item}[align left]
            \begin{description}
            	\item[Daten] von mehreren Sensoren/Verfahren.
            	\item[Kombination] zu einer einzigen Punktwolke
			\end{description}

}
\section{Eigenschaften}
\frame{\tableofcontents[currentsection]}
%!TEX root = ../../main.tex


\subsection{Struktur}
\frame{
	\frametitle{Was gehört dazu?}

	\begin{description}
		\item [Feature Selection] Welche Punkte sind \textit{repräsentativ} für das Objekt/die Szene?
		\item [1. Reference Frame/Axis] Welche \textit{Referenz} wird für die Nachbarschaft gewählt
		\item [2. Histogramm/Signatur] Welche \textit{Berechnungsvorschrift} wird ausgeführt?
		\item [3. Matching-Phase] Wie werden berechnete Deskriptoren verglichen?
	\end{description}


	    \begin{figure}
        \centering
        \begin{subfigure}[b]{0.35\textwidth}
            \includegraphics[width=\textwidth]{topics/eigenschaften/hist.png}
            \caption{Histogramm}
            \label{fig:hist}
        \end{subfigure}
        ~~~
        \begin{subfigure}[b]{0.35\textwidth}
            \includegraphics[width=\textwidth]{topics/intro/n3d.png}
            \caption{3D-Nachbarschaft}
            \label{fig:ctKnut}
        \end{subfigure}
    \end{figure}


}
\subsection{Schwierigkeiten}
\frame{
	\frametitle{Was sind die Schwierigkeiten?}

	\begin{flushleft}
	\setbeamertemplate{description item}[align left]
	\begin{description}
		\item[Geometrische Transformationen] Rotation und Translation sollen möglichst geringen Einfluss auf das Ergebnis haben.
		\item[Skalierung] Ein in der größe skaliertes Objekt soll die selbe Deskription erhalten.
		\item[Punktdichte] Unterschiedliche Sensoren und Blickwinkel führen zu Schwankungen.
		\item[Clutter] Viele Objekte, ein Durcheinander
		\item[Verdeckung] Objekte verdecken andere
		\item[Vorzeichen] Eindeutigkeit des Referenzsystems.
	\end{description}
	\end{flushleft}
}
\section{Auswahl}
\frame{ \tableofcontents[currentsection] }
\subsection{Spin Images}
\frame{
	\frametitle{Spin Images}
}
\subsection{Point Signatures}
\frame{
	\frametitle{Point Signatures}
}
\subsection{Exponential Mapping}
\frame{
	\frametitle{Exponential Mapping}
}
\subsection{SHOT}
\frame{
	\frametitle{Signatures of Histogramms of Orientations}
}
\section{Vergleich}
\frame{ \tableofcontents[currentsection] }
\frame{
	\frametitle{Im Experiment}
	\centering
		\textbf{Versuch 1}~~~~~~~~~~~~~\textbf{Versuch 2}~~~~~~~~~~~~~\textbf{Versuch 3}
		\begin{columns}[c]
    	\begin{column}{3cm}
    	\centering
    	
    	\begin{itemize}
    		\item 6 Modelle aus Stanford 3D Scanning Repository
    		\item Mit Modellen 45 Szenen erzeugt
    	\end{itemize}

     	\end{column}
    	\begin{column}{3cm}
    	\centering
    	
    	\end{column}
    	\begin{column}{3cm}
    	\centering
    	
    	\end{column}
    \end{columns}

}


%\subsection{Bewertung}
\frame{
	\frametitle{False-Posives?}
	\centering
		\includegraphics[height=0.8\textheight]{topics/vergleich/pr.jpg}
}


%%Precision recall
\frame{
	\frametitle{Die Maße: Precision und Recall}
	\begin{columns}[c]
    	\begin{column}{5cm}
     		Wie viele der gelieferten Muster sind relevant?
     		$$Precision = \frac{|TP|}{|TP|+|FP|}$$
     		Wie viele der richtigen Muster wurden geliefert? 
     		$$ Recall\frac{|TP|}{|TP|+|FN|}$$
     	\end{column}
    	\begin{column}{5cm}
     		\includegraphics[height=0.8\textheight]{topics/vergleich/pr.jpg}
    	\end{column}
    \end{columns}
}

\frame{
	\frametitle{Ergebnis: Versuch 1}
	\centering
	\includegraphics[height=0.8\textheight]{topics/vergleich/v1.png}
}

\frame{
	\frametitle{Ergebnis: Versuch 2}
	\centering
	\includegraphics[width=\textwidth]{topics/vergleich/subSampled.png}
}

\frame{
	\frametitle{Ergebnis: Versuch 3}
	\centering
	\includegraphics[width=\textwidth]{topics/vergleich/v3.png}
}
\frame{\frametitle{Vielen Dank für die 
Aufmerksamkeit}}
\end{document}
