\documentclass[12pt]{article}
\usepackage[utf8]{inputenc}
\usepackage[ngerman]{babel}
\usepackage{amsmath,amssymb}
\usepackage{graphicx}
\usepackage{url}
\usepackage{textcomp}
\usepackage{hyperref}
\usepackage{listings}
\usepackage{caption}
\usepackage{subcaption}
\usepackage{multimedia}
\usepackage{color}
\usepackage{float}
\usepackage{geometry}


\begin{document}
\section{Gliederung}
\subsection{Einführung}




Zu Beginn werden durch Anwendungsbeispiele die folgenden Fragen geklärt:\\
\begin{itemize}
	\item Was versteht man unter \textbf{Surface-Matching/Deskriptoren}
	\item Zur Lösung von welchen Problemen wird es eingesetzt( Rekonstruktion/Objekterkennung )
	\item Klärung der nötigen Begriffe (descriptor,signature,histogram,support,reference frame,reference axis,...)
\end{itemize}
\subsection{Was bisher geschah}
Es wird ein Überblick über die bisherige Entwicklung gegeben.
\begin{itemize}
	\item Welche Ansätze wurden bisher verfolgt? (signature/histogram)
	\item Wo liegen die Schwierigkeiten (clutter,noise,translation,rotation)
	\item Wie kann man Deskriptoren bewerten? (invariance,robustness)
\end{itemize}
\subsection{Fortschritt}
Welche Probleme sehen Tombari et. al, wie lösen Sie diese
\begin{itemize}
	\item Einfluss des Reference-Frames auf den Deskriptor
	\item Ein oder mehrere Reference-Frames (uniqueness/ambiguity)
	\item Ein guter Reference-Frame durch eigenvalue-decomposition und das Erzeugen von Eindeutigkeit durch Berechnung eines Vorzeichens.
\end{itemize}
\subsection{Signaturen von Histogrammen \textbf{SHOT}}
\begin{itemize}
	\item{ 
		Von 2D nach 3D: Warum funktioniert SIFT und wie kann man diese Prinzipien auf dreidimensionale Probleme anwenden}
	\item{
		Trotz Ableitungen robust gegenüber Rauschen? 
	} 
	\item {
		Kombination von Histogrammen über Winkel zwischen Punkt- und Umfeldnormalenvektoren als Deskriptor an der Schnittstelle zwischen Signatur und Histogramm
	}
	\item{
		Wahl des Koordinatensystems für die Signatur
	}
	\item{
		Robustheit gegenüber Schwankungen in der Punktdichte
	}
\end{itemize}
\subsection{Validierung im Experiment}
\begin{itemize}
	\item Wie wird Wiederholbarkeit garantiert?
	\item Wie werden faire Wettbewerbsbedingungen garantiert?
	\item Ist SHOT besser als die gewählten Gegner?
	\item Wenn ja, wie viel in welchen Bereichen.
\end{itemize}
\section{Literatur}
Neben dem zentralen Paper von Tombari et al. werden auch die in der Literaturliste der Veröffentlichung genannten Werke verwendet, hier aber nicht nochmal einzeln aufgeführt.
\begin{description}
	\item[Unique Signatures of Histograms for Local Surface
Description]{
	by Federico Tombari, Samuele Salti, and Luigi Di Stefano
} 
	\item[Three-Dimensional Computer Vision]{
	 by Olivier D. Faugeras
	}
	\item[Multiple View Geometry in Computer Vision]{
	 by Richard Hartley and Andrew Zisserman
	}
	\item[Introductory Techniques For 3D Computer Vision]{
emanuele trucco and alessandro verri
	}

\end{description}
\end{document}